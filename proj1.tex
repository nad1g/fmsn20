\documentclass[a4paper,10pt]{article}
\usepackage{amsmath}

%defines
\def\bY{{\bf Y}}
\def\bE{{\bf E}}
\def\bV{{\bf V}}
\def\bC{{\bf C}}
\def\b1{{\bf 1}}
\def\blambda{{\boldsymbol \lambda}}
\def\bgamma{{\boldsymbol \gamma}}
\def\bnu{{\boldsymbol \nu}}

%opening
\title{Home Assignment - 1}
\author{Santhosh Nadig}

\begin{document}

\maketitle

\section{Introduction}

\section{Theory}
Let $\bY_k$ represent the $n$ known observations of spatial data and $\bY_u$ represent the $m$ data points that needs to be estimated. For ordinary Kriging ($\mu = \b1 \beta$), the optimal predictions are given by
\begin{align*}
 \hat \bY_u &= \b1_u \hat \beta + \Sigma_{uk} \Sigma_{kk}^{-1}\left( \bY_k - \b1_k \hat \beta \right) \\
 \hat \beta &= \left( \b1_k^T \Sigma_{kk}^{-1} \b1_k \right)^{-1} \b1_k^T \Sigma_{kk}^{-1} \bY_k,
\end{align*}
where
\begin{align*}
 &\begin{bmatrix}
  \bY_k \\
  \bY_u
 \end{bmatrix} \in
 \mathbf{N} \left( 
 \begin{bmatrix}
    \b1_k \beta \\
   \b1_u \beta
  \end{bmatrix},
  \begin{bmatrix}
   \Sigma_{kk} & \Sigma_{ku} \\
   \Sigma_{uk} & \Sigma_{uu}
  \end{bmatrix}
  \right).
\end{align*}

Firstly, we show that the predictions are linear in the observations, i.e., $\hat \bY_u = \blambda^T \bY_k$ for some $\blambda$. We may re-write $\hat \beta$ as
\begin{align*}
 \hat \beta &= \bgamma^T \bY_k \\
 \bgamma &= \left( \left( \b1_k^T \Sigma_{kk}^{-1} \b1_k \right)^{-1} \b1_k^T \Sigma_{kk}^{-1} \right)^T.
\end{align*}
We note that the vector $\bgamma^T$ is of dimensions $1 \times n$, and therefore, $\hat \beta$ is a scalar. Substituting the above in the expression for $\hat \bY_u$, we obtain
\begin{align*}
 \hat \bY_u &= \b1_u \bgamma^T \bY_k + \Sigma_{uk} \Sigma_{kk}^{-1} \bY_k -  \Sigma_{uk} \Sigma_{kk}^{-1}  \b1_k \bgamma^T \bY_k \\
      &= \left( \b1_u \bgamma^T + \Sigma_{uk} \Sigma_{kk}^{-1} -  \Sigma_{uk} \Sigma_{kk}^{-1}  \b1_k \bgamma^T \right) \bY_k \\
      &= \blambda^T \bY_k,
\end{align*}
where $\blambda = \left( \b1_u \bgamma^T + \Sigma_{uk} \Sigma_{kk}^{-1} -  \Sigma_{uk} \Sigma_{kk}^{-1}  \b1_k \bgamma^T \right)^T$.

Secondly, we show that the predictions are unbiased, i.e., $\bE(\hat \bY_u) = \bE(\bY_u)$. The expected value of $\hat \beta$ is given by
\begin{align*}
 \bE \left( \hat \beta \right) &= \left( \b1_k^T \Sigma_{kk}^{-1} \b1_k \right)^{-1} \b1_k^T \Sigma_{kk}^{-1} ~ \bE (\bY_k) \\
  &= \left( \b1_k^T \Sigma_{kk}^{-1} \b1_k \right)^{-1} \b1_k^T \Sigma_{kk}^{-1} ~ \b1_k \beta \\
  &= \beta.
\end{align*}
Therefore,
\begin{align*}
 \bE(\hat \bY_u) &=  \b1_u \bE\left(\hat \beta \right) + \Sigma_{uk} \Sigma_{kk}^{-1}\left( \bE(\bY_k) - \b1_k \bE \left(\hat \beta \right)\right) \\
 &= \b1_u \beta + \Sigma_{uk} \Sigma_{kk}^{-1}\left( \b1_k \beta - \b1_k \beta \right) \\
 &= \b1_u \beta.
\end{align*}

Finally, consider a second unbiased predictor of $\bY_u$, given by $\tilde \bY_u = (\blambda + \bnu)^T \bY_k$. From the unbiasedness criterion, we have that $\bE(\tilde \bY_u) = \bE(\hat \bY_u)$. Therefore, $(\blambda + \bnu)^T \bE(\bY_k) = \blambda^T \bE(\bY_k)$, implying $\bnu^T \b1_k = 0$. Furthermore, consider the variance of the second predictor
\begin{align*}
 \bV(\tilde \bY_u - \bY_u) &= \bV(\bnu^T \bY_k + (\blambda^T \bY_k - \bY_u))\\
 &= \bV(\bnu^T \bY_k) + \bV(\hat \bY_u - \bY_u) + 2\bC(\bnu^T \bY_k, \hat \bY_u - \bY_u) \\
 &\ge \bV(\hat \bY_u - \bY_u) + 2\bC(\bnu^T \bY_k, \hat \bY_u - \bY_u)\\
 &= \bV(\hat \bY_u - \bY_u) + 2\bnu^T \bC(\bY_k, \hat \bY_u) - 2 \bnu^T \bC(\bY_k, \bY_u) \\
 &= \bV(\hat \bY_u - \bY_u) + 2\bnu^T \left(  \Sigma_{kk} \blambda - \Sigma_{uk}\right)
\end{align*}
Thus, if we can choose $\blambda$ such that $\bnu^T \left(  \Sigma_{kk} \blambda - \Sigma_{uk}\right) = 0$, then, we have that $\bV(\tilde \bY_u - \bY_u) \ge \bV(\hat \bY_u - \bY_u)$ for any unbiased predictor $\tilde \bY_u$.

\section{Swedish Temparature Reconstruction}
The data consists of average temperature during June 2005 measured at 250 stations across Sweden. In addition, the latitude, longitude, elevation and distance to both the Swedish coast and to any coastline are provided too. Figure~  shows the plot of average temperature, elevation and distance to any coast. The average temperatures are higher in the South (lower latitudes) and towards the coast, as expected.
\subsection{Least-Squares}

\subsection{Universal Kriging}

\section{Conclusions}

\end{document}
